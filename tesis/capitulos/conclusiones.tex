\chapter{Conclusiones}

Es necesario abordar los problemas vistos a lo largo de este trabajo, tanto de la parte de los protocolos como del crecimiento de la Web. Si bien existen diversos enfoques para mejorar los tiempos de respuesta de la carga de los sitios, ofrecen una soluci'on dada ciertas caracter'isticas del contexto en el que se est'an utilizando. SPDY es un protocolo que muestra que se pueden mejorar ciertos aspectos sin tener un gran impacto en la implementaci'on. Pero, su rendimiento se encuentra relacionado directamente con el contexto en que se est'a utilizando, cuestiones como el ancho de banda, p'erdida de paquetes, tama'no de los recursos, cantidad de recursos peticionados, afectan los tiempos de respuesta.

En este trabajo se propone y se desarrolla un proxy adaptativo capaz de decidir qu'e protocolo utilizar, dado un contexto determinado, con el cual se optimicen los tiempos de carga de las p'aginas. Se basa en las caracter'isticas propias de los protocolos en los que se ven beneficiados o no. Con el fin de evaluar la performance, se dise'na y se ejecuta un experimento para obtener el tiempo de carga de ciertas p'aginas y poder comparar el proxy propuesto con un proxy convencional.

Los resultados son favorables en gran parte de los sitios que aceptan los 3 protocolos. La adecuaci'on de los protocolos al contexto en el que se est'a operando, es decir, utilizar el protocolo que se vea m'as beneficiado con las condiciones dadas, provee una leve mejora en el rendimiento general de la carga de los sitios. Esta mejora es m'as significativa en el caso de las conexiones que se realizan con HTTPS.

En cuanto a los sitios del ranking de Alexa a la fecha actual, la mayor'ia utiliza HTTPS para operar, pero solo el 20\% soportan SPDY. Con lo cual, no existe esta opci'on para poder mejorar el rendimiento. Es importante, al menos ofrecer el protocolo alternativo, ya que uno podr'ia, dado el contexto, tomar la decisi'on por uno u otro dadas las condiciones del contexto.

SPDY sent'o las bases del nuevo HTTP 2.0 que, si bien todav'ia no est'a implementado, est'a en v'ias de implementarse en un futuro cercano. La utilizaci'on de SPDY, servir'ia para poder experimentar con el protocolo y poder adecuar los sitios para que su rendimiento sea 'optimo, cuesti'on importante previendo una futura migraci'on a HTTP 2.0.

Otra cuesti'on importante es que, la utilizaci'on de un proxy que pueda manejar SPDY u otros protocolos en un futuro, brinda la posibilidad de aprovechar las caracter'isticas de dichos protocolos, sin necesidad de que el cliente conectado al proxy, tenga un navegador que soporte dichos protocolos. Tambi'en, que el proxy act'ue de MITM, ofrece la posibilidad de utilizar herramientas como una cach'e web, para mejorar la performance, en sitios en los que no se puede utilizar esta opci'on (por ejemplo, HTTPS).

\section{Trabajos futuros}

A partir del desarrollo de este trabajo se detectaron m'ultiples oportunidades de expansi'on y nuevas l'ineas de trabajo. Como continuidad del mismo se propone, extraer el algoritmo de selecci'on de protocolo e implementarlo en un cliente directo, tal como un navegador. Para as'i tener la ventaja de no tener ning'un intermediario y realizar la conexi'on directamente con el servidor final, ahorr'andose las conexiones extras.

Por otro lado, implementar el proxy como reverso, brindar'ia la posibilidad de ofrecer SPDY para servidores que todav'ia no lo soportan.

Agregar la funcionalidad de que el cliente pueda conectarse con SPDY como protocolo, brinda la posibilidad de tener conectados 2 proxies m'as conectados entre s'i. 'Util por ejemplo para tenerlos funcionando en ubicaciones geogr'aficas diferentes y armar un CDN que se comunica con SPDY, es decir, por una sola conexi'on TCP.

En un futuro cuando HTTP 2.0 ya se encuentre en funcionamiento, extender el proxy para aceptar este protocolo y sumarlo al 'arbol de decisi'on, da la posibilidad de poder probar su rendimiento.