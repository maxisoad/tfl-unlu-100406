\chapter{Introducci'on}

El principal protocolo utilizado en la Web es HTTP\footnote{Hypertext Transfer Protocol.} \citep{rfcHTTP1.1}, tuvo su primera versi'on en mayo de 1996, culminando en 1999 con el est'andar actual que es HTTP 1.1. Es un protocolo sin estado, lo que conlleva a que sea necesario realizar una conexi'on nueva por cada recurso que se necesite en un sitio. Por ejemplo, si una p'agina posee 5 im'agenes y un archivo de Javascript\footnote{Lenguaje de programaci'on interpretado que se utiliza principalmente del lado del cliente en el navegador.}, se van a realizar 7 conexiones para obtener toda la p'agina completa, la primera para obtener la p'agina, y una vez que el navegador lee el contenido de la misma, realiza el resto de las peticiones para obtener todos los elementos.

Los sitios web de la actualidad difieren de las p'aginas de hace m'as de 10 a'nos, tanto en tama'no como en cantidad de recursos. Este fen'omeno, hace que el protocolo ya no tenga el mismo rendimiento que en 'epocas anteriores. A pesar del avance de la tecnolog'ia en cuanto a mejoras en las velocidades de los enlaces de red, poseer gran ancho de banda no es el 'unico factor que interviene en la performance de la carga de las p'aginas.
Google propuso un protocolo llamado SPDY\footnote{SPeeDY.} como soluci'on a los problemas actuales, busca mejorar la performance de HTTP siendo transparente su implementaci'on.
Se ver'a en los cap'itulos siguientes, los problemas que presentan los protocolos, y bajo qu'e circunstancias se ven beneficiados o perjudicados en cuanto a su performance.

\section{Motivaci'on}

La IETF\footnote{Internet Engineering Task Force - http://www.ietf.org} es una organizaci'on internacional que regula las propuestas y los est'andares de Internet, conocidos como RFC\footnote{Request For Comments}. Un grupo de trabajo que pertenece a la IETF llamado \textit{HTTPbis}\footnote{Hypertext Transfer Protocol Bis.} se encuentra desarrollando el nuevo protocolo HTTP 2.0 \citep{http2} que es la evoluci'on de HTTP. Este protocolo, toma como punto de partida a SPDY y se contin'uan realizando cambios y mejoras. Por esto, es importante estudiar el rendimiento de los protocolos, en especial el de SPDY que es la base de la evoluci'on del protocolo m'as utilizado en Internet.

\section{Objetivos}

En este trabajo, se presenta la problem'atica, se propone, y desarrolla un servidor proxy con un algoritmo adaptativo que determina seg'un el perfil de cada sitio que m'etodo (\textsc{http}, \textsc{https} 'o \textsc{spdy}) elegir para mejorar la performance, es decir, que decisi'on ser'ia la 'optima.

\section{Organizaci'on del trabajo}

En el Cap'itulo 2 se conoce el desarrollo que tuvo la web a lo largo de estos a'nos, se observa el crecimiento  de Internet y de los sitios

En el Cap'itulo 3, se avanza sobre diversos conceptos de los protocolos que se utilizan actualmente, as'i tambi'en como sus debilidades. 

En el Cap'itulo 4, se estudian las problem'aticas asociadas a la web y ciertas t'ecnicas de optimizaci'on. 

En el Cap'itulo 5, se ve el concepto de proxy, necesario para comprender el desarrollo en s'i, tambi'en se introducen ciertas cuestiones de cach'e web en el Cap'itulo 6. Ya con toda la informaci'on necesaria, en el Cap'itulo 7, se ve el desarrollo del proxy propuesto, con sus caracter'isticas y funcionalidades. 

Hacia el final, en el Cap'itulo 8, con el fin de evaluar las prestaciones del proxy en un ambiente real, se realiza un experimento y se comentan sus resultados. En el 'ultimo Cap'itulo se exponen las conclusiones finales y los trabajos futuros.

El glosario, la bibliograf'ia y los anexos se encuentran al final de este trabajo. 

No existe un cap'itulo dedicado a trabajos relacionados como tal, pero estos pueden encontrarse a lo largo de los cap'itulos, especialmente en el cap'itulo del desarrolo de la web (Cap'itulo \ref{desarrolloWeb}) y de la problem'atica de la misma (Cap'itulo \ref{problematica}). 