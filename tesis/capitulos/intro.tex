\chapter{Introducci'on}

El protocolo HTTP tuvo su primera versi'on en mayo de 1996, culminando en 1999 con el est'andar actual que es HTTP 1.1. Es un protocolo sin estado, lo que conlleva a que sea necesario realizar una conexi'on nueva por cada recurso que se necesite en un sitio.

Los sitios web de la actualidad difieren de las p'aginas de hace m'as de 10 a'nos, tanto en tama'no como en cantidad de recursos. Este fen'omeno, hace que el protocolo ya no tenga el mismo rendimiento que en 'epocas anteriores. A pesar del avance de la tecnolog'ia en cuanto a mejoras en las velocidades de los enlaces de red, poseer gran ancho de banda no es el 'unico factor que interviene en la performance de la carga de las p'aginas.
Google propuso un protocolo llamado SPDY, como soluci'on a los problemas actuales, busca mejorar la performance de HTTP siendo transparente su implementaci'on.
Veremos en los cap'itulos siguientes, los problemas que presentan los protocolos, y bajo qu'e circunstancias se ven beneficiados o perjudicados en cuanto a su performance.

Se busca como objetivo de este trabajo, desarrollar un proxy con un algoritmo inteligente que pueda determinar seg'un el perfil del sitio que m'etodo (\textsc{http}, \textsc{https} 'o \textsc{spdy}) elegir para mejorar la performance, es decir, que decisi'on ser'ia la 'optima.

El trabajo se encuentra organizado de la siguiente manera, veremos el desarrollo que tuvo la Web a lo largo de estos a'nos, avanzaremos sobre diversos conceptos de los Protocolos que se utilizan actualmente, as'i tambi'en como sus debilidades. Estudiaremos las problem'aticas asociadas a la Web y ciertas t'ecnicas de optimizaci'on. Conoceremos el concepto de Proxy, necesario para comprender el desarrollo en s'i, tambi'en se introducir'an ciertas cuestiones de Cach'e Web. Ya con toda la informaci'on necesaria, nos adentraremos en el desarrollo del Proxy propuesto, con sus caracter'isticas y funcionalidades. Hacia el final, con el fin de probar el Proxy en un ambiente real, veremos un experimento, sus resultados y las conclusiones finales. El glosario, la bibliograf'ia y los anexos se encuentran al final de este trabajo. No existe un cap'itulo dedicado a Trabajos Relacionados como tal, pero estos pueden encontrarse a lo largo de los cap'itulos, especialmente en el cap'itulo de Protocolos (Cap'itulo \ref{protocolos}).
