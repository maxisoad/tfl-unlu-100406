\chapter{Problem'atica}

Debido al crecimiento de Internet visto en el C'apitulo \ref{desarrolloWeb} y a los problemas que presenta el protocolo HTTP en su implementaci'on (visto en el C'apitulo \ref{protocolos}), se busca mejorar los tiempos de carga de los sitios web. 

%PONER PORQUE SE BUSCA MEJORAR LOS TIEMPOS DE RESPUESTA.

\section{T'ecnicas de optimizaci'on}

Es importante conocer d'onde es que el usuario pasa el tiempo esperando en la carga de un sitio web. Seg'un el estudio de Steve Souders en su libro \citep{highPerformanceWebSites}, el cliente tarda menos del 20\% para obtener el documento HTML, y el tiempo restante para recibir el resto de los componentes del sitio. Es importante enfocarse en el 80\%, 90\% restante, ya que el tiempo no se desperdicia en descargar el documento HTML ni en el procesamiento que realiza el servidor antes de enviarnos la petici'on. Esto se resume en la ''Regla de Oro de la Performance'' de dicho libro:

\begin{quote}

\textit{Solo el 10-20\% del tiempo de respuesta del usuario es consumido descargando el documento HTML. El otro 80-90\% se consume descargando todos los componentes de la p'agina.}

\end{quote}

Plantea 14 reglas para la optimizaci'on de los sitios que se describen a continuaci'on.

\begin{enumerate}

\item \textbf{Minimizar la cantidad de peticiones.}

La idea b'asica es eliminar las peticiones al servidor, esto se hace utilizando varias t'ecnicas enumeradas a continuaci'on.

	\begin{enumerate}
	\item Mapa de Im'agenes - Permite asociar m'ultiples 'areas en una imagen para que se puedan cliquear y realizar cierta acci'on (el ejemplo m'as b'asico es el de navegar un hiperv'inculo). En vez de utilizar m'ultiples im'agenes para realizar un men'u por ejemplo, se utilizar'ia una sola mapeada. De esta manera se ahorran las peticiones de las im'agenes por 1 sola petici'on.
	\item CSS Sprites - Permite combinar v'arias im'agenes en una sola, luego en el sitio, se define que parte de la imagen combinada desea mostrarse. Por ejemplo, en el sitio de Google cuando se realiza una b'usqueda, se utiliza el Sprite que se ve en la Figura \ref{css_sprites}.
		\begin{figure}[htbp]
		\begin{center}
			\includegraphics[width=150px]{img/css_sprites}
			\caption{\small Sprite del sitio www.google.com}
		\end{center}
		\label{css_sprites}
		\end{figure}

	A simple vista se ven 44 im'agenes combinadas. La utilizaci'on de este Sprite reducir'ia todas estas peticiones a 1 sola.
	\item Im'agenes Inline - http://tools.ietf.org/html/rfc2397
	\item Combinar Scripts y Hojas de Estilo
	\end{enumerate}	
\item \textbf{Utilizar un CDN}
\item \textbf{A'nadir un Header Expires}
\item \textbf{Comprimir los componentes (Gzip)}
\item \textbf{CSS en el Head del HTML}
\item \textbf{Scripts abajo de todo en el HTML}
\item \textbf{Evitar expresiones en CSS}
\item \textbf{Utilizar JavaScript y CSS de manera externa (no inline)}
\item \textbf{Reducir las b'usquedas de DNS (keep-alive y pocos dominios)}
\item \textbf{Minificar el JavaScript}
\item \textbf{Evitar redirecciones}
\item \textbf{Remover Scripts duplicados}
\item \textbf{Configurar Etags}
\item \textbf{Hacer que Ajax pueda ser Cacheado (BUSCAR OTRA PALABRA)}

\end{enumerate}