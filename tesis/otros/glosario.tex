\newglossaryentry{url}{
	name= {URL - Localizador de Recursos Uniforme},
	description={Secuencia de caracteres, de acuerdo a un formato mod'elico y est'andar, que se usa para nombrar recursos en Internet para su localizaci'on o identificaci'on, como por ejemplo documentos textuales, im'agenes, v�deos, presentaciones digitales, etc. El formato general de un URL es:
esquema://m'aquina/directorio/archivo}
}

\newglossaryentry{latencia}{
	name={Latencia},
	description={Retraso en la transmisi'on de datos desde un punto a otro}
}

\newglossaryentry{ancho}{
	name={Ancho de Banda},
	description={Es la longitud, medida en Hz, del rango de frecuencias en el que se concentra la mayor parte de la potencia de la se'nal. Tambi'en puede referirse a la tasa media de transferencia de datos exitosa a trav'es de una v'ia de comunicaci'on. Puede conocerse como capacidad del canal de comunicaci'on}
}

\newglossaryentry{mime}{
	name={MIME - Multipurpose Internet Mail Extensions},
	description={Son una serie de convenciones o especificaciones dirigidas al intercambio a trav'es de Internet de todo tipo de archivos (texto, audio, video, etc.) de forma transparente para el usuario. Una parte importante del MIME est'a dedicada a mejorar las posibilidades de transferencia de texto en distintos idiomas y alfabetos}
}

\newglossaryentry{rfc}{
	name={RFC - Request For Comments},
	description={Petici'on de Comentarios. Son una serie de notas sobre Internet, y sobre sistemas que se conectan a internet, que comenzaron a publicarse en 1969. Cada una de ellas individualmente es un documento cuyo contenido es una propuesta oficial para un nuevo protocolo de la red Internet, que se explica con todo detalle para que en caso de ser aceptado pueda ser implementado sin ambiguedades}
}

\newglossaryentry{netcraft}{
	name={Netcraft},
	description={Es una compa'n'ia de servicios de Internet basada en Bath, Inglaterra. Netcraft ofrece an'alisis de cuota de mercado de servidores y alojamiento web, incluyendo la detecci'on del tipo de servidor web y de sistema operativo}
}

\newglossaryentry{html}{
	name={HTML - HyperText Markup Language},
	description={Lenguaje de Marcado de Hipertexto. Es un est'andar que sirve de referencia para la elaboraci'on de p'aginas web en sus diferentes versiones, define una estructura b'asica y un c'odigo (denominado c�digo HTML) para la definici'on de contenido de una p'agina web}
}

\newglossaryentry{pdf}{
	name={PDF - Portable Document Format},
	description={Formato de Documento Port'atil. Es un formato de almacenamiento de documentos digitales independiente de plataformas de software o hardware. Este formato es de tipo compuesto (imagen vectorial, mapa de bits y texto). Fue inicialmente desarrollado por la empresa Adobe Systems, oficialmente lanzado como un est'andar abierto el 1 de julio de 2008 y publicado por la Organizaci'on Internacional de Estandarizaci'on como ISO 32000-1}
}

\newglossaryentry{uri}{
	name={URI - Universal Resource Identifier},
	description={Identificador Universal de Recursos. Es una cadena de caracteres corta que identifica inequ'ivocamente un recurso. Ver \citep{rfcSintaxURI}}
}

\newglossaryentry{gmt}{
	name={GMT - Greenwich Mean Time},
	description={Identificador Universal de Recursos. Es un est'andar de tiempo que originalmente se refer'ia al tiempo solar medio en el Real Observatorio de Greenwich, en Greenwich, cerca de Londres, Inglaterra, que en 1884 fue elegido por la Conferencia Internacional del Meridiano como el primer meridiano}
}

\newglossaryentry{dominio}{
	name={Dominio},
	description={Un dominio de Internet es una red de identificaci'on asociada a un grupo de dispositivos o equipos conectados a Internet. El prop�sito principal de los nombres de dominio en Internet y del sistema de nombres de dominio (DNS), es traducir las direcciones IP de cada nodo activo en la red, a t'erminos memorizables y f'aciles de encontrar}
}


\newglossaryentry{dns}{
	name={DNS - Domain Name System},
	description={Sistema de nombres de dominio. Es un sistema de nomenclatura jer'arquica para computadoras, servicios o cualquier recurso conectado a Internet o a una red privada. Este sistema asocia informaci'on variada con nombres de dominios asignado a cada uno de los participantes. Su funci'on m'as importante, es traducir (resolver) nombres inteligibles para las personas en identificadores binarios asociados con los equipos conectados a la red, esto con el prop'osito de poder localizar y direccionar estos equipos mundialmente}
}

\newglossaryentry{iso}{
	name={ISO - International Organization for Standardization},
	description={Es el organismo encargado de promover el desarrollo de normas internacionales de fabricaci'on (tanto de productos como de servicios), comercio y comunicaci'on para todas las ramas industriales a excepci�n de la el'ectrica y la electr'onica. Su funci'on principal es la de buscar la estandarizaci'on de normas de productos y seguridad para las empresas u organizaciones (p'ublicas o privadas) a nivel internacional}
}

\newglossaryentry{wan}{
	name={WAN - Wide Area Network},
	description={Es una red de computadoras que abarca varias ubicaciones f'isicas, proveyendo servicio a una zona, un pa'is, incluso varios continentes. Es cualquier red que une varias redes locales, por lo que sus miembros no est'an todos en una misma ubicaci�n f'isica}
}

\newglossaryentry{lan}{
	name={LAN - Local Area Network},
	description={Es una red que interconecta computadoras dentro de un 'area limitada. Se encuentran dentro de una 'area geogr'afica peque'na}
}

\newglossaryentry{3g}{
	name={Telefon'ia M'ovil 3G},
	description={Es la abreviaci'on de tercera generaci'on de transmisi'on de voz y datos a trav'es de telefon'ia m'ovil mediante UMTS (servicio universal de telecomunicaciones m'oviles)}
}

\newglossaryentry{script}{
	name={Script},
	description={Es un programa usualmente simple, que por lo regular se almacena en un archivo de texto plano. Los script son casi siempre interpretados, pero no todo programa interpretado es considerado un script. El uso habitual de los scripts es realizar diversas tareas como combinar componentes, interactuar con el sistema operativo o con el usuario}
}